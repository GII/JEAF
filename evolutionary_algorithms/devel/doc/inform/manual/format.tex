%%%%%%%%%%%%%%%%%%%%%%%%%%%%%%%%%%%%%%%%%%%%%%%%%%%%%%%%%%%%%%%%%%%%%%%%%%
%     This is format.tex file needed for the dmathesis.cls file.  You    %
%  have to  put this file in the same directory with your thesis files.  %
%                Written by M. Imran 2001/06/18                          % 
%                 No Copyright for this file                             % 
%                 Save your time and enjoy it                            % 
%                                                                        % 
%%%%%%%%%%%%%%%%%%%%%%%%%%%%%%%%%%%%%%%%%%%%%%%%%%%%%%%%%%%%%%%%%%%%%%%%%%
%%%%%  Put packages you want to use here and 'fancyhdr' is required   %%%%
%%%%%%%%%%%%%%%%%%%%%%%%%%%%%%%%%%%%%%%%%%%%%%%%%%%%%%%%%%%%%%%%%%%%%%%%%%
\usepackage{fancyhdr}
\usepackage{epsfig}
\usepackage{cite}
\usepackage{graphicx}
\usepackage{amsmath}
\usepackage{theorem}
\usepackage{amssymb}
\usepackage{latexsym}
\usepackage{epic}
\usepackage{setspace}

%\usepackage{anysize}
%%%%%%%%%%%%%%%%%%%%%%%%%%%%%%%%%%%%%%%%%%%%%%%%%%%%%%%%%%%%%%%%%%%%%%%%%%
%%%%%                 Set line spacing = 1.5 here                   %%%%%%
%%%%%%%%%%%%%%%%%%%%%%%%%%%%%%%%%%%%%%%%%%%%%%%%%%%%%%%%%%%%%%%%%%%%%%%%%%
\renewcommand{\baselinestretch}{1.5}
\setlength{\parskip}{6pt}

%\marginsize{3cm}{2cm}{3cm}{3cm}

%%%%%%%%%%%%%%%%%%%%%%%%%%%%%%%%%%%%%%%%%%%%%%%%%%%%%%%%%%%%%%%%%%%%%%%%%%
%%%%%                      Your fancy heading                       %%%%%%
%%%%% For the final copy you need to remove '\bfseries\today' below %%%%%%
%%%%%%%%%%%%%%%%%%%%%%%%%%%%%%%%%%%%%%%%%%%%%%%%%%%%%%%%%%%%%%%%%%%%%%%%%%
\pagestyle{fancy}
%\headheight 16pt
\renewcommand{\chaptermark}[1]{\markright{\chaptername\ \thechapter.\ #1}}
\renewcommand{\sectionmark}[1]{\markright{\thesection.\ #1}{}}
\lhead[\fancyplain{}{\bfseries\thepage}]%
      {\fancyplain{}{\bfseries\rightmark}}
\chead[\fancyplain{}{}]%
      {\fancyplain{}{}}
\rhead[\fancyplain{}{\bfseries\rightmark}]%
      {\fancyplain{}{\bfseries\thepage}}
\lfoot[\fancyplain{}{}]%
      {\fancyplain{}{}}
\cfoot[\fancyplain{}{}]%
      {\fancyplain{}{}}
\rfoot[\fancyplain{}{}]%
      {\fancyplain{}{}}
%%%%%%%%%%%%%%%%%%%%%%%%%%%%%%%%%%%%%%%%%%%%%%%%%%%%%%%%%%%%%%%%%%%%%%%%%%
%%%%%%%%%%%%Here you set the space between the main text%%%%%%%%%%%%%%%%%%
%%%%%%%%%%%%%%%%%%%and the start of the footnote%%%%%%%%%%%%%%%%%%%%%%%%%%
%%%%%%%%%%%%%%%%%%%%%%%%%%%%%%%%%%%%%%%%%%%%%%%%%%%%%%%%%%%%%%%%%%%%%%%%%%
\addtolength{\skip\footins}{5mm}
%%%%%%%%%%%%%%%%%%%%%%%%%%%%%%%%%%%%%%%%%%%%%%%%%%%%%%%%%%%%%%%%%%%%%%%%%%
%%%%%      Define new counter so you can have the equation           %%%%%
%%%%%    number 4.2.1a for example, this a gift from J.F.Blowey      %%%%%
%%%%%%%%%%%%%%%%%%%%%%%%%%%%%%%%%%%%%%%%%%%%%%%%%%%%%%%%%%%%%%%%%%%%%%%%%%
\newcounter{ind}
\def\eqlabon{
\setcounter{ind}{\value{equation}}\addtocounter{ind}{1}
\setcounter{equation}{0}
\renewcommand{\theequation}{\arabic{chapter}%
         .\arabic{section}.\arabic{ind}\alph{equation}}}
\def\eqlaboff{
\renewcommand{\theequation}{\arabic{chapter}%
         .\arabic{section}.\arabic{equation}}
\setcounter{equation}{\value{ind}}}
%%%%%%%%%%%%%%%%%%%%%%%%%%%%%%%%%%%%%%%%%%%%%%%%%%%%%%%%%%%%%%%%%%%%%%%%%%
%%%%%%%%%%%%           New theorem you want to use              %%%%%%%%%%
%%%%%%%%%%%%%%%%%%%%%%%%%%%%%%%%%%%%%%%%%%%%%%%%%%%%%%%%%%%%%%%%%%%%%%%%%%
{\theorembodyfont{\rmfamily}\newtheorem{Pro}{{\textbf Proposition}}[section]}
{\theorembodyfont{\rmfamily}\newtheorem{The}{{\textbf Theorem}}[section]}
{\theorembodyfont{\rmfamily}\newtheorem{Def}[The]{{\textbf Definition}}}
{\theorembodyfont{\rmfamily}\newtheorem{Cor}[The]{{\textbf Corollary}}}
{\theorembodyfont{\rmfamily}\newtheorem{Lem}[The]{{\textbf Lemma}}}
{\theorembodyfont{\rmfamily}\newtheorem{Exp}{{\textbf Example}}[section]}
\def\remark{\textbf{Remark}:}
\def\remarks{\textbf{Remarks}:}
\def\bproof{\textbf{Proof}: }
\def\eproof{\hfill$\Box$}
%%%%%%%%%%%%%%%%%%%%%%%%%%%%%%%%%%%%%%%%%%%%%%%%%%%%%%%%%%%%%%%%%%%%%%%%%%
%%%%%%%    Bold font in math mode, a gift from J.F.Blowey       %%%%%%%%%%
%%%%%%%%%%%%%%%%%%%%%%%%%%%%%%%%%%%%%%%%%%%%%%%%%%%%%%%%%%%%%%%%%%%%%%%%%%
\def\bv#1{\mbox{\boldmath$#1$}}
%%%%%%%%%%%%%%%%%%%%%%%%%%%%%%%%%%%%%%%%%%%%%%%%%%%%%%%%%%%%%%%%%%%%%%%%%%
%%%%%%%        New symbol which is not defined in Latex         %%%%%%%%%%
%%%%%%%                 a gift from J.F.Blowey                  %%%%%%%%%%
%%%%%%%%%%%%%%%%%%%%%%%%%%%%%%%%%%%%%%%%%%%%%%%%%%%%%%%%%%%%%%%%%%%%%%%%%%
% The Mean INTegral
% to be used in displaystyle
\def\mint{\textstyle\mints\displaystyle}
% to be used in textstyle
\def\mints{\int\!\!\!\!\!\!{\rm-}\ }
%
% The Mean SUM
% to be used in displaystyle
\def\msum{\textstyle\msums\displaystyle}
% to be used in textstyle
\def\msums{\sum\!\!\!\!\!\!\!{\rm-}\ }
%%%%%%%%%%%%%%%%%%%%%%%%%%%%%%%%%%%%%%%%%%%%%%%%%%%%%%%%%%%%%%%%%%%%%%%%%%
%%%%%%%%%%            Define your short cut here              %%%%%%%%%%%%
%%%%%%%%%%%%%%%%%%%%%%%%%%%%%%%%%%%%%%%%%%%%%%%%%%%%%%%%%%%%%%%%%%%%%%%%%%
\def\poincare{Poincar\'e }
\def\holder{H\"older }



















